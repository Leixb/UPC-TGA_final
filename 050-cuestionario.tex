%! TEX root = **/010-main.tex
% vim: spell spelllang=es:

\begin{pregunta}{Cuestionario. Pregunta Doble. Las 5 últimas cuestiones (f-j) se contestan con 1 frase. Las 5
primeras (a-e) requieren una explicación más elaborada.}

\begin{enumerate}[label=(\alph*)]
    \item ¿Cual es la diferencia entre un fragmento y un pixel?

        Un pixel es un elemento final de la imagen, un fragmento es una parte de
        una primitiva que se corresponde a un pixel. La principal diferencia es
        que un fragmento puede tener mas información asociada que un pixel y que
        un fragmento esta asociado a un pixel pero un pixel puede generar-se a
        partir de la combinación de varios fragmentos, este seria el caso cuando
        se combinan 1 o más fragmentos transparentes para formar un pixel
        (\emph{alpha blending}) o si se aplica algún tipo de
        \emph{anti-aliasing}.

    \item ¿Qué hace el rasterizador?

        Genera los fragmentos (grupos de pixels que forman parte de un mismo
        triangulo) a partir de la información de las ecuaciones de
        triángulos de las primitivas. Estos fragmentos rasterizados se
        corresponden a pixels de la imagen final y serán procesados por el
        \emph{fragment shader}.

    \item Siempre decimos que en una GPU ocultamos la latencia con memoria. ¿Puedes explicar cómo lo hacemos?

        Mediante paralelismo y cambios de contexto de ejecución. Mientras una
        parte de la memoria se esta copiando (latencia) se cambia el contexto de
        ejecución para realizar otras tareas. De modo similar se pueden a
        empezar a ejecutar \emph{kernels} a medida que se transfieren los datos
        de memoria necesarios. No hace falta esperar a terminar todas las copias
        para empezar la ejecución. \cite{volkov_understanding_nodate}.

    \item ¿Qué significa que una GPU tenga los shaders unificados?

        Que las unidades de procesamiento de la GPU pueden ejecutar
        \emph{shaders} de cualquier tipo. Esto permite poder aprovechar al
        máximo las unidades de procesado de la GPU ya que si necesitamos muchos
        más \emph{vertex shaders} que \emph{fragment shaders} no estamos
        limitados.

    \item ¿Para qué sirve el Z buffer? ¿Cómo lo hace?

        El \emph{Z buffer} determina la posición de los fragmentos en el eje
        \emph{Z} (perpendicular al plano cámara) y se usa para determinar que
        elementos quedan delante de otros. Para ello se calcula la distancia de
        cada fragmento respecto a los valores \emph{zNear} y \emph{zFar}
        (limites de renderizado en profundidad) y se guarda el valor de la
        \emph{Z} más cercana el el \emph{Z buffer} (tamaño igual a la imagen
        final). Si se encuentra un nuevo pixel mas lejano se descarta, si esta
        mas cerca se actualiza.

        \pagebreak
%%
    \item Define en una frase CLIPPING

        Operación realizada durante el procesado de primitivas del pipeline
        gráfico en la que se descartan primitivas que quedan fuera del campo de
        visión eliminando completamente los vértices del triangulo o partiéndolo
        en triángulos mas pequeños.

    \item Define en una frase CULLING

        Operación realizada durante el procesado de primitivas del pipeline
        gráfico en la que se descartan primitivas que quedan ocluidos por otras
        primitivas usando el \emph{z-buffer}, al igual que con el \emph{CLIPPING}
        se puede descartar la primitiva entera o partir-la en nuevas primitivas
        si la oclusión es parcial.

    \item ¿Qué es REYES?

        \emph{Renders Everything You Ever Saw} es un algoritmo de renderizado de
        gráficos foto-realista que no sigue el método tradicional actual y usa
        muchos micro-polígonos. Esta enfocado a la creación de contenido
        multimedia (desarrollado por la actual \emph{Pixar}).
        \cite{shua_rispec_2011}

    \item ¿Qué es el Aspect Ratio?

        Es la relación entre el ancho y el alto de una imagen.

    \item Muchas veces hablamos de GPGPU, ¿que quieren decir estas siglas?

        \emph{General Purpose Graphics Procesing Unit}: una tarjeta gráfica que
        se puede programar para realizar tareas de propósito general (no solo
        procesado de gráficos).

\end{enumerate}

\end{pregunta}
