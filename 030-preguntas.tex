%! TEX root = **/010-main.tex
% vim: spell spelllang=es:

% Antes de empezar el examen, leed atentamente los siguientes comentarios:
%
% - Esto es un examen y los exámenes se hacen de forma individual.
%
% - La fecha límite para entregar el examen (y el proyecto) es el lunes 21 de
%   junio de 2021 a las 8:00.
%
% - He montado una “práctica” en el racó para hacer la entrega del examen.
%
% - El examen que entreguéis ha de ser un PDF.
%
% - Hemos calculado que el tiempo necesario para realizar el examen son unas
%   20-30 horas de trabajo. No lo dejéis para el final.
%
% - Teniendo en cuenta que tenéis mucho tiempo para hacer el examen, una buena
% redacción y ortografía se tendrá en cuenta en la evaluación.
%
% - No os limitéis a la información que hay en las transparencias del curso, hay
%   que buscar en otras fuentes.
%
% - Además del examen tenéis que entregar un anexo, en el que, para cada
%   pregunta, indiquéis las referencias, páginas web, etc. que habéis consultado
%   durante el examen.
%
% - El examen consta de:
%   - 6 preguntas estándar (2 hojas por pregunta, aprox.  1000 palabras por pregunta).
%   - 1 pregunta doble, es cómo un pequeño trabajo (4 hojas, aprox. 2000 palabras)

\newcommand{\sep}{\hrule}

\newcommand{\pregunta}[1]{
    \pagebreak
    \begin{enumerate}[resume]
        \item #1
    \end{enumerate}
}

\pregunta{Describe el pipeline gráfico tradicional}\sep{}

\pregunta{Dada la siguiente rutina escrita en C:}

\begin{minted}{c}
void Examen21(float mA[N][M], float mB[N][M], float vC[N], float vD[M]) {
  int i, j;
  for (i=0; i<N; i++)
    for (j=0; j<M; j++)
         mA[i][j] = mA[i][j]*vC[i] - mB[i][j]*vD[j] + mA[i][0]*mB[7][j];
}
\end{minted}

Escribid 3 versiones del kernel CUDA que resuelva el mismo problema:

\begin{enumerate}[label=(\alph*)]
    \item En la primera versión cada thread se va a ocupar de 1 columna de la matriz resultado.
    \item En la segunda versión cada thread se va a ocupar de 1 fila de la matriz resultado.
    \item En la última versión cada thread se va a ocupar de 1 elemento de la matriz resultado.
\end{enumerate}

Escribid los kernels CUDA para cada versión, así como la invocación
correspondiente. Tened en cuenta que como máximo podéis utilizar 1024 threads
por bloque y que las variables N y M pueden tener cualquier valor (p.e. N =
1237, M = 2311, suponed que N, M > 1024).

\vspace{1em} \sep{}

\pregunta{Disponemos de una tarjeta gráfica con 2 GPUs. En esta tarjeta queremos
    correr un juego interactivo 3D (que utiliza OpenGL u otra API similar). Si
    estuvierais diseñando el driver de la API gráfica, ¿cómo distribuirías el
    trabajo entre las 2 GPUs para maximizar el rendimiento?  ¿Qué información
    hay que enviar a cada tarjeta? ¿Han de sincronizarse/comunicarse las 2 GPUs?
    ¿Cómo pueden hacerlo? Os ayudará tener en mente cómo funciona el pipeline
gráfico} \sep{}

\pregunta{Una de las herramientas que utilizamos en CUDA son los eventos (event en
CUDA). ¿Para qué sirven? ¿Cómo se utilizan? Pon ejemplos de uso.} \sep{}

\pregunta{Si queremos utilizar GPUs para cálculo de propósito general (GPGPU)
    puedes escoger entre CUDA, OpenCL y OpenACC. Describe las ventajas e
    inconvenientes de cada alternativa.  Además, se pueden combinar. ¿Qué
posibilidades ofrece combinar OpenACC con CUDA o OpenCL?} \sep{}

\pregunta{Hablando de texturas, ¿qué filtros existen?, ¿puedes describirlos? ¿qué
implicaciones tienen en el diseño de la GPU?} \sep{}
