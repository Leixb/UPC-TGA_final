%! TEX root = **/010-main.tex
% vim: spell spelllang=es:

\begin{pregunta}{Pregunta doble. Explica qué es el raytracing, cómo funciona, y cómo se
    implementa en las nuevas tarjetas de Nvidia, las RTX. Os ayudará que esta
pregunta la encaréis cómo si fuera un pequeño trabajo.} \sep{}

\subsection*{¿Qué es el raytracing?}

El raytracing es un técnica renderizaci\'on de gráficos basada en la proyección de
rayos a través de la escena. En un sistema de renderizado se calcula como los objetos
de la escena se proyectan en la imagen final, es decir para cada objeto se busca
a que pixels de la imagen final afecta. En raytracing se hace lo contrario:
para cada pixel de la imagen final, se proyecta un rayo partiendo de la cámara
que \emph{busca} los objetos que intersecan. La ventaja de raytracing viene de
la capacidad de calcular rayos recursivamente y ponderar-los para obtener un
color final del pixel basado en la interacción con el resto de objetos de la
escena. Como consiguiente con raytracing se obtienen imágenes mucho mas
foto-realistas debido a que considera las sombras, reflejos, refracción,
ínter-reflejos y otros efectos que no se pueden calcular con rasterizado.
Un claro ejemplo son las sombras, con raytracing se pueden calcular de modo
fiable y realista pero con rasterizado se tienen que calcular como una textura a
parte y combinar-lo en la escena.

El principal problema del raytracing es que requiere muchísima potencia
computacional y por consiguiente no se puede usar para renderizado en tiempo
real. Su uso hasta hace poco estaba limitado a simulaciones usadas para
cinematografía en la que cada frame puede tardar varios minutos en calcular-se.

% El raytracing es una técnica de renderizaci\'on de gráficos basado en el trazado
% del camino que hacen los rayos de luz en la escena. A diferencia del sistema de
% rasterización estándar, el raytracing es un modelo de luz global, es decir tiene
% en consideración todas las fuentes de luz de la escena no solo las fuentes
% principales. Esto incluye los efectos de difusión de color entre objetos.
% Como consiguiente es capaz de simular efectos ópticos como
% reflexión, refracción, dispersión y sombras entre otros.

% La capacidad de la técnica de raytracing para calcular efectos ópticos de manera
% realista hace que se obtengan imágenes mucho más foto-realistas que mediante el
% rasterizado convencional (que para simular estos efectos ópticos tiene que
% recurrir a técnicas alternativas como texturas, \emph{stencils} \dots). Sin
% embargo el coste computacional del raytracing es mucho mayor y hasta hace poco
% no se podía aplicar en tiempo real y su uso estaba limitado al ámbito
% cinematográfico.


\subsection*{Funcionamiento}

En términos mas básicos, el raytracing es un algoritmo mas simple que el del
renderizado ya que con el mismo método se puede calcular todo (no hacen falta
varios procesos para calcular sombras, reflejos, \dots). \cite{leopold_english_2017}

\subsubsection*{Ray-casting}

Para cada pixel de la imagen final, se traza un rayo partiendo de la posición de
la cámara que pase por ese pixel. Se calcula con que objetos de la escena ínterseca
este rayo. Una vez se obtenido el punto de intersección, se traza un rayo desde
este nuevo punto hacia las fuentes de luz y se calcula si hay algún otro objeto
ocluyendo la luz. Con la información de la luz y del materia podemos saber que
color tendrá este punto. Este algoritmo de dos etapas es el método más básico de
raytracing y produce resultados similares a un renderizado simple.

\subsubsection*{Reflejos y refracciones}

La parte interesante del raytracing es cuando se aplica recursividad al
algoritmo anterior. Esta vez, para cada colisión no solo trazamos un rayo hacia
la fuente de luz sino que también calculamos otro rayo reflectado y otro
refractado (si el materia es refractante) y aplicamos otra vez el mismo
algoritmo. Ahora tenemos una escena en el que a parte de sombras tenemos
reflejos y refracciones.

\subsubsection*{Refracción}

Adicionalmente se puede calcular la difusión (como el propio material refleja
luz en varias direcciones aunque sea mate), para ello en cada colisión
calculamos varios rayos en diversas direcciones. Dependiendo del tipo de
material estos rayos irán en todas direcciones o estarán predominantemente hacia
una sola. Por ejemplo, un espejo perfecto solo lanzara un rayo en el angulo
opuesto a la incidencia de la luz y un objeto completamente mate lanzará hacia
todos los lados, en medio tendríamos todo un espectro de varios materiales
lustrosos que lanzarán varios rayos en dirección similar a la del espejo mas o
menos esparcidos dependiendo de su lustro.


\subsection*{Detalles sobre el coste computacional}

El problema con todas estas técnicas es que ya no hablamos de un solo rayo por
pixel sino de un número bastante considerable. Para tener una buena imagen se
deben usar alrededor de 5000 rayos para cada pixel (Esto sirve también como una
técnica de anti-aliasing ya que los rayos se hacen empezar en varios puntos del
pixel). Estos valores son los usados por estudios como \emph{Pixar} en el
renderizado de sus películas y ofrecen una calidad de imagen excepcional.
Desafortunadamente no se puede realizar a tiempo real. Este tipo de raytracing
en el que se calculan tantos rayos se llama \emph{path-tracing}.

\subsubsection*{Bounding Volume Hierarchy}

El sistema de \emph{Bounding Volume Hierarchy} (\emph{BVH}) encapsula los
objetos en \emph{Bounding boxes} en un árbol donde la caja raíz contiene toda
la escena y las ramas contienen grupos de objetos hasta llegar a los objetos en
si. Este árbol permite poder calcular la intersección de un rayo en una escena
de $N$ objetos en $\mathcal{O}(\log{}n)$ ya que permite una búsqueda dicotómica
en el espacio.

\subsubsection*{Requisitos de memoria}

Otro de los problemas que tiene el raytracing es que se accede a puntos de la
memoria de modo no coherente. A diferencia del renderizado en el que un pixel
solo accede a un punto o varios puntos de una misma textura, en raytracing los
rayos pueden llevarnos a requerir acceder a información sobre cualquier objeto
de la imagen por lo que es difícil tener un sistema que aproveche bien la cache.


\subsection*{Nvidia RTX}

En 2018, \emph{Nvidia} presentó su nueva arquitectura de tarjetas gráficas llamada \emph{Turing}.
Estas nuevas tarjetas, a con el nombre \emph{RTX}, tenían la novedad de incluir
los nuevos \emph{RT cores} (\emph{RayTracing cores}). Estas unidades de
procesamiento están especializadas en los procesos de cálculo necesarios para
raytracing. Con estas nuevas tarjetas es posible generar gráficos con raytracing
en tiempo real. \cite{noauthor_introducing_2018}

En el apartado anterior he hablado de las complicaciones de aplicar raytracing
en tiempo real debido al coste computacional. Sin embargo, las nuevas tarjetas
\emph{RTX} de \emph{Nvidia} permiten raytracing en tiempo real. Para
conseguir-lo \emph{Nvidia} implemento en sus tarjetas dos nuevos sistemas de
computación especializados: los \emph{RT cores} y los \emph{Tensor cores}. Este
hardware especializado permite reducir parte de los problemas inherentes del
raytracing. En las siguientes secciones hablaré en concreto de que técnicas se
aplican para conseguir poder usar raytracing en tiempo real.

\subsubsection*{RT cores}

Los \emph{Raytracing cores} se encargan de acelerar el proceso de proyectar de los
rayos. Están especializados en dos operaciones: atravesar el \emph{BVH} y
calcular la intersección de un rayo con un triángulo. Usando esta tecnología es
posible calcular muy rápidamente las intersecciones de los rayos.

\subsection*{Tensor cores}

Los \emph{Tensor cores} vienen de la familia de tarjetas gráficas de
\emph{Nvidia} especializadas en \emph{Machine Learning}. Estos cores están
optimizados para cálculos usados en redes neuronales. A priori se podría pensar
que no hay una relación entre estos y la renderización por raytracing, pero
\emph{Nvidia} usa un sistema basado en redes neuronales para reducir el ruido de
las imágenes generadas con raytracing.

Cuando se genera una imagen usando raytracing, la calidad del resultado depende
del numero de rayos usado para cada pixel. Si se usa un solo rayo por pixel, es
posible que pixels contiguos el usando rayos de difusión aleatorio en las
superficies hagan que queden de colores muy distintos, produciendo una imagen
con ruido. Es por eso que en cinematografía se usan alrededor de 5000 rayos para
cada pixel. Las nuevas tarjetas de \emph{Nvidia} son capaces de usar muy pocos
rayos y aplicar un algoritmo de reducción de ruido asistido por redes neuronales
para eliminar ese ruido y generar la imagen final.

\subsection*{Usos de raytracing}

En la mayoría de casos no se usa raytracing para renderizar toda la escena, sino
que se usa una combinación de rasterización tradicional con raytracing en áreas
de interés (materiales con propiedades reflectantes, refractantes \dots). Este
sistema es distinto al sistema de \emph{path-tracing} que renderiza toda la
escena usando rayos. Pero para videojuegos la diferencia entre usar \emph{RTX}
en algunos juegos es notable.

\paragraph{Iluminación global} la mayoría de juegos compatibles con \emph{RTX}
usan el raytracing para el cálculo de la iluminación, solo con el uso de
\emph{RTX} para la iluminación podemos obtener sombras suaves, reflejos de
superficie y ínter-reflexiones (difusión de luz por parte de los objetos).
Sin raytracing se tienen que calcular las sombras, refracciones y reflejos como
escenas a parte y aplicar-las de modo similar a texturas.

\paragraph{Oclusión de ambiente} (\emph{Ambient oclusion}) permite añadir
volumen a los objetos aplicándoles sombras basadas en objetos cercanos
(basándose solo en la iluminación de ambiente). Se consigue trazando rayos
alrededor de cada punto de la escena y calculando cuantas colisiones reciben
estos rayos en un radio determinado. Este efecto se puede aproximar
sin raytracing usando el \emph{depth buffer} pero el resultado es muy distinto.

\paragraph{Profundidad de campo} Con raytracing se puede aplicar un efecto de
profundidad de campo mas realista que afecte incluso a las superficies
refractadas y reflejadas.

\paragraph{Distorsión de movimiento} Aplicando raytracing con muestras en
posiciones ligeramente distintas del modelo en la escena se puede fácilmente
conseguir un efecto de \emph{motion blur} sin necesidad de recurrir a otras
técnicas.

\paragraph{Efectos atmosféricos} Al tener información sobre el trazado de los
rayos de luz en la escena se pueden añadir efectos atmosféricos como la
dispersión de la luz.

\paragraph{Cáustica} la cáustica es el efecto producido por la envolvente de los
rayos de luz refractados o reflejados por una superficie curva cobre otra
superficie. Por ejemplo cuando un rayo de sol atraviesa una copa de vino o los
patrones que se generan por la refracción en las olas del mar sobre el fondo del
mar. No existen métodos para aproximar cáusticas de modo general usando
rasterizado, solo se pueden usar simulaciones aproximadas para causticas conocidas
como las de las olas del mar.

\subsection*{Conclusión}

Los avances en el campo del raytracing en los últimos años han permitido llegar
a tener raytracing en tiempo real en tarjetas para consumidores. La combinación
del \emph{machine learning} como ayuda en procesos de renderizado es un campo
prometedor que podría permitir mejorar drásticamente el modo en que se diseñan
juegos. Un ejemplo es el estudio que vimos en clase, realizado por \emph{Intel
labs} en el que se mejoraba la calidad de los graficos de \emph{GTA V} a través
de \emph{ML} \cite{richter_enhancing_2021}.

\nocite{haines_ray_2019,haines_ray_2020}



\end{pregunta}
