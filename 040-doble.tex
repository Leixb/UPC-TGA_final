%! TEX root = **/010-main.tex
% vim: spell spelllang=es:

\begin{pregunta}{Pregunta doble. Explica qué es el raytracing, cómo funciona, y cómo se
    implementa en las nuevas tarjetas de Nvidia, las RTX. Os ayudará que esta
pregunta la encaréis cómo si fuera un pequeño trabajo.} \sep{}

\subsection*{¿Qué es el raytracing?}

El raytracing es una técnica de renderizaci\'on de gráficos basado en el trazado
del camino que hacen los rayos de luz en la escena. A diferencia del sistema de
rasterización estándar, el raytracing es un modelo de luz global, es decir tiene
en consideración todas las fuentes de luz de la escena no solo la fuente
principal. Como consiguiente es capaz de simular efectos ópticos como
reflexión, refracción, dispersión y sombras entre otros.

La capacidad de la técnica de raytracing para calcular efectos ópticos de manera
realista hace que se obtengan imágenes mucho más foto-realistas que mediante el
rasterizado convencional (que para simular estos efectos ópticos tiene que
recurrir a técnicas alternativas como texturas, \emph{stencils} \dots). Sin
embargo el coste computacional del raytracing es mucho mayor y hasta hace poco
no se podía aplicar en tiempo real y su uso estaba limitado al ámbito
cinematográfico.


\subsection*{Funcionamiento}

Hay diversos algoritmos de \emph{raytracing} que varían en algunos puntos (Des
de donde empezar los rayos, control de profundidad \dots) Pero todos se basan en
el mismo concepto: trazar rayos de luz desde (o hasta) la cámara.
% BVH

\cite{leopold_english_2017}

\subsection*{Nvidia RTX}

En 2018, \emph{Nvidia} presentó su nueva arquitectura de tarjetas gráficas llamada \emph{Turing}.
Estas nuevas tarjetas, a con el nombre \emph{RTX}, tenían la novedad de incluir
los nuevos \emph{RT cores} (\emph{RayTracing cores}). Estas unidades de
procesamiento están especializadas en los procesos de cálculo necesarios para
raytracing. Con estas nuevas tarjetas es posible generar gráficos con raytracing
en tiempo real. \cite{noauthor_introducing_2018}

\end{pregunta}
